
%%%%%%%%%%%%%%%%%%%%%% Feynman diagram for GMSBqqqqGG

\documentclass{article}

\usepackage{feynmp-auto}

%%%%%% Custom commands
\def\chiz{$\mathrm{\widetilde{\chi}^0_1}$}
\def\chitz{$\mathrm{\widetilde{\chi}^0_{2}}$}
\def\chipm{$\mathrm{\widetilde{\chi}^\pm_{1}}$}
\def\chimp{$\mathrm{\widetilde{\chi}^\mp_{1}}$}
\def\sGlu{$\mathrm{\widetilde{g}}$}
\def\sQ{$\mathrm{\widetilde{q}}$}
\def\sQb{$\mathrm{\overline{\widetilde{q}}}$}
\def\sBot{$\mathrm{\widetilde{b}_1}$}
\def\sBotb{$\mathrm{\overline{\widetilde{b}}_1}$}
\def\sTop{$\mathrm{\widetilde{t}_1}$}
\def\sTopb{$\mathrm{\overline{\widetilde{t}}_1}$}
\def\sTopt{$\mathrm{\widetilde{t}_2}$}
\def\sToptb{$\mathrm{\overline{\widetilde{t}}_2}$}
\def\sGra{$\mathrm{\widetilde{G}}$}
\def\PH{$\mathrm{H}$}
\def\PZ{$\mathrm{Z}$}
\def\PW{$\mathrm{W^\pm}$}
\def\PV{$\mathrm{V}$}
\def\qprime{$\mathrm{q}^\prime$}
\def\Ph{$\mathrm{h}$}
\def\sLep{$\widetilde{\ell}$}
\def\sNu{$\widetilde{\nu}$}
\def\sTau{$\widetilde{\tau}$}
\newcommand{\anti}[1]{$\mathrm{\overline{#1}}$}




%%%%%%%%%%%%%%%%%%%%%%%%%%% Document %%%%%%%%%%%%%%%%%%%%%%%%%%%
\begin{document}
\thispagestyle{empty}


%%%%%%%% THE NAME OF THE fmffile HAS TO BE ``Feynman<filename>'' TO USE compile.py %%%%%%%%%%%%%%%
\begin{fmffile}{FeynmanGMSBqqqqGG}
\parbox{300mm}{

\begin{fmfgraph*}(180,90) %\fmfpen{thick}

  \fmfset{arrow_len}{cm}\fmfset{arrow_ang}{0}

  %%%%%%%%%%%% Specifying number of inputs/outputs

  \fmfleftn{i}{2}

  \fmfforce{0.w,0.95h}{i1}
  \fmfforce{0.w,0.05h}{i2}


  \fmfbottom{o10,o11,o11a,o1,o2,o3}
  \fmfright{o4,o5}
  \fmftop{o12,o13,o13a,o8,o7,o6}
  %%%%%%%%%%%% Incoming protons (one line)
  \fmf{fermion,  tension=2., lab=p, label.side=right}{v1,i2}
  \fmf{fermion,  tension=2., lab=p, label.side=left}{v1,i1}

    \fmfforce{0.43w,0.75h}{v2}
    \fmfforce{0.62w,0.75h}{v3}
    \fmfforce{0.82w,0.75h}{v4}
    \fmfforce{w,0.75h}{o5}

    \fmfforce{0.43w,0.25h}{v5}
    \fmfforce{0.62w,0.25h}{v6}
    \fmfforce{0.82w,0.25h}{v7}
    \fmfforce{w,0.25h}{o4}



  %%%%%%%%%%%% Produced SUSY particles
  \fmf{gluon, label=\sGlu, label.side=left, label.dist=+8}{v1,v2}
  \fmf{gluon, label=\sGlu, label.side=right, label.dist=+12}{v1,v5}

  \fmf{fermion}{v1,v2}
  \fmf{fermion}{v1,v5}
  %%%%%%%%%%%%% Decays and vertex circles

  %%% Top vertex
  \fmf{dashes, label=\sQ,label.dist=+3, label.side=left}{v5,v6}
  \fmf{dots,label=\chiz,label.dist=-14}{v6,v7}
      \fmflabel{\anti{q}}{o1}
      \fmflabel{q}{o2}
           \fmflabel{\sGra}{o4}
           \fmflabel{$\gamma$}{o3}
           \fmf{fermion}{o1,v5}
           \fmf{fermion}{o2,v6}
           \fmf{dots}{o4,v7}
           \fmf{photon}{o3,v7}

  %%% Bottom vertex
  \fmf{dashes, label=\sQb, label.dist=+3, label.side=right}{v2,v3}
  \fmf{dots,label=\chiz,label.dist=3}{v3,v4}
   \fmflabel{$\gamma$}{o6}
   \fmflabel{\sGra}{o5}
   \fmflabel{\anti{q}}{o7}
   \fmflabel{q}{o8}
           \fmf{fermion}{o8,v2}
           \fmf{fermion}{o7,v3}
           \fmf{dots}{o5,v4}
           \fmf{photon}{o6,v4}

  %% Vertex circles
  \fmfdot{v2,v3,v4,v5,v6,v7}

  %%%%%%%%%%%% Additional lines on incoming protons and blob
             
  %%%%%%%%%%%% Additional lines on incoming protons and blob
  \fmffreeze
  \renewcommand{\P}[3]{\fmfi{plain}{%
      vpath(__#1,__#2) shifted (thick*(#3))}}
  \P{i1}{v1}{2.,-0}
  \P{i1}{v1}{-2,0}
  \P{i2}{v1}{2.,0}
  \P{i2}{v1}{-2.,-0}
  \fmfv{decor.shape=circle,decor.filled=30, decor.size=.12w}{v1}



\end{fmfgraph*}

}
\end{fmffile}

\end{document}
