
%%%%%%%%%%%%%%%%%%%%%% Feynman diagram for VVtoVV

\documentclass{article}

\input{shared/header.tex} 

%%%%%%%%%%%%%%%%%%%%%%%%%%% Document %%%%%%%%%%%%%%%%%%%%%%%%%%%
\begin{document}
\thispagestyle{empty}


%%%%%%%% THE NAME OF THE fmffile HAS TO BE ``Feynman<filename>'' TO USE compile.py %%%%%%%%%%%%%%%
\begin{fmffile}{FeynmanVVtoVV}
\parbox{300mm}{

\begin{fmfgraph*}(180,90) %\fmfpen{thick}

  \fmfset{arrow_len}{cm}\fmfset{arrow_ang}{0}

  %%%%%%%%%%%% Specifying number of inputs/outputs
  \fmfleftn{i}{2}
  \fmfrightn{o}{4}
  \fmflabel{}{i1}
  \fmflabel{}{i2}

  %%%%%%%%%%%% Incoming quarks
  \fmf{fermion, tension=2, lab=q, label.side=right}{v1,i1}
  \fmf{fermion, tension=2, lab=q, label.side=left}{v3,i2}

  %%%%%%%%%%%% Produced boson and quarks
  \fmf{fermion}{v1,o1}
  \fmf{fermion}{v3,o4}
  \fmf{photon, label=\PV, label.dist=2, label.side=left, tension=1}{v1,v2}
  \fmf{photon, label=\PV, label.dist=2, label.side=right, tension=1}{v3,v2}
  %%%%%%%%%%%%% Decays and vertex circles

  \fmf{photon}{v2,o2}
  \fmf{photon}{v2,o3}
  \fmflabel{\qprime}{o1}
  \fmflabel{\qprime}{o4}
  \fmflabel{\PV}{o2}
  \fmflabel{\PV}{o3}

  %% Vertex circles
  \fmfv{decor.shape=circle,decor.filled=100, decor.size=.03w}{v1}
  \fmfv{decor.shape=circle,decor.filled=100, decor.size=.03w}{v3}
  \fmfv{decor.shape=diamond,decor.filled=100, decor.size=.08w, foreground=green}{v2}

  %%%%%%%%%%%% Additional lines on incoming protons and blob
  %\input{shared/protons.tex}


\end{fmfgraph*}

}
\end{fmffile}

\end{document}
