
%%%%%%%%%%%%%%%%%%%%%% Feynman diagram for VVtoVV

\documentclass{article}

\usepackage{feynmp-auto}

%%%%%% Custom commands
\def\chiz{$\mathrm{\widetilde{\chi}^0_1}$}
\def\chitz{$\mathrm{\widetilde{\chi}^0_{2}}$}
\def\chipm{$\mathrm{\widetilde{\chi}^\pm_{1}}$}
\def\chimp{$\mathrm{\widetilde{\chi}^\mp_{1}}$}
\def\sGlu{$\mathrm{\widetilde{g}}$}
\def\sQ{$\mathrm{\widetilde{q}}$}
\def\sQb{$\mathrm{\overline{\widetilde{q}}}$}
\def\sBot{$\mathrm{\widetilde{b}_1}$}
\def\sBotb{$\mathrm{\overline{\widetilde{b}}_1}$}
\def\sTop{$\mathrm{\widetilde{t}_1}$}
\def\sTopb{$\mathrm{\overline{\widetilde{t}}_1}$}
\def\sTopt{$\mathrm{\widetilde{t}_2}$}
\def\sToptb{$\mathrm{\overline{\widetilde{t}}_2}$}
\def\sGra{$\mathrm{\widetilde{G}}$}
\def\PH{$\mathrm{H}$}
\def\PZ{$\mathrm{Z}$}
\def\PW{$\mathrm{W^\pm}$}
\def\PV{$\mathrm{V}$}
\def\qprime{$\mathrm{q}^\prime$}
\def\Ph{$\mathrm{h}$}
\def\sLep{$\widetilde{\ell}$}
\def\sNu{$\widetilde{\nu}$}
\def\sTau{$\widetilde{\tau}$}
\newcommand{\anti}[1]{$\mathrm{\overline{#1}}$}

 

%%%%%%%%%%%%%%%%%%%%%%%%%%% Document %%%%%%%%%%%%%%%%%%%%%%%%%%%
\begin{document}
\thispagestyle{empty}


%%%%%%%% THE NAME OF THE fmffile HAS TO BE ``Feynman<filename>'' TO USE compile.py %%%%%%%%%%%%%%%
\begin{fmffile}{FeynmanVVtoVV}
\parbox{300mm}{

\begin{fmfgraph*}(180,90) %\fmfpen{thick}

  \fmfset{arrow_len}{cm}\fmfset{arrow_ang}{0}

  %%%%%%%%%%%% Specifying number of inputs/outputs
  \fmfleftn{i}{2}
  \fmfrightn{o}{4}
  \fmflabel{}{i1}
  \fmflabel{}{i2}

  %%%%%%%%%%%% Incoming quarks
  \fmf{fermion, tension=2, lab=q, label.side=right}{v1,i1}
  \fmf{fermion, tension=2, lab=q, label.side=left}{v3,i2}

  %%%%%%%%%%%% Produced boson and quarks
  \fmf{fermion}{v1,o1}
  \fmf{fermion}{v3,o4}
  \fmf{photon, label=\PV, label.dist=2, label.side=left, tension=1}{v1,v2}
  \fmf{photon, label=\PV, label.dist=2, label.side=right, tension=1}{v3,v2}
  %%%%%%%%%%%%% Decays and vertex circles

  \fmf{photon}{v2,o2}
  \fmf{photon}{v2,o3}
  \fmflabel{\qprime}{o1}
  \fmflabel{\qprime}{o4}
  \fmflabel{\PV}{o2}
  \fmflabel{\PV}{o3}

  %% Vertex circles
  \fmfv{decor.shape=circle,decor.filled=100, decor.size=.03w}{v1}
  \fmfv{decor.shape=circle,decor.filled=100, decor.size=.03w}{v3}
  \fmfv{decor.shape=diamond,decor.filled=100, decor.size=.08w, foreground=green}{v2}

  %%%%%%%%%%%% Additional lines on incoming protons and blob
  %           
  %%%%%%%%%%%% Additional lines on incoming protons and blob
  \fmffreeze
  \renewcommand{\P}[3]{\fmfi{plain}{%
      vpath(__#1,__#2) shifted (thick*(#3))}}
  \P{i1}{v1}{2.,-0}
  \P{i1}{v1}{-2,0}
  \P{i2}{v1}{2.,0}
  \P{i2}{v1}{-2.,-0}
  \fmfv{decor.shape=circle,decor.filled=30, decor.size=.12w}{v1}



\end{fmfgraph*}

}
\end{fmffile}

\end{document}
