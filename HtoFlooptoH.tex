
%%%%%%%%%%%%%%%%%%%%%% Feynman diagram for VtoVV

\documentclass{article}

\input{shared/header.tex} 

%%%%%%%%%%%%%%%%%%%%%%%%%%% Document %%%%%%%%%%%%%%%%%%%%%%%%%%%
\begin{document}
\thispagestyle{empty}


%%%%%%%% THE NAME OF THE fmffile HAS TO BE ``Feynman<filename>'' TO USE compile.py %%%%%%%%%%%%%%%
\begin{fmffile}{FeynmanHtoFlooptoH}
\parbox{300mm}{

\begin{fmfgraph*}(120,90) %\fmfpen{thick}

  \fmfset{arrow_len}{cm}\fmfset{arrow_ang}{0}

  %%incoming and outgoing
  \fmfleft{i}
  \fmfright{o}

  %%vertex
  \fmfdot{v1}
  \fmfdot{v2}

  \fmf{dashes, label=$\Phi$}{i,v1}
  \fmf{dashes, label=$\Phi$}{v2,o}

  %%loop
  \fmf{fermion, left, label=f}{v1,v2,v1}


  %%%%%%%%%%%% Additional lines on incoming protons and blob
  %\input{shared/protons.tex}

\end{fmfgraph*}

}
\end{fmffile}

\end{document}
